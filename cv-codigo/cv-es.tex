\documentclass[11pt, a4paper]{article}
% --- Configuración básica del documento ---
\usepackage[utf8]{inputenc}      % Soporte para caracteres UTF-8 (tildes, ñ)
\usepackage[T1]{fontenc}         % Mejor renderizado de fuentes
\usepackage[spanish]{babel}      % Idioma español (hyphenation, textos automáticos)
\usepackage[margin=0.75in]{geometry} % Márgenes reducidos para aprovechar espacio
\usepackage{xcolor}              % Soporte para colores personalizados
\usepackage{parskip}             % Espaciado entre párrafos (no sangría)
\usepackage{fontawesome}         % Iconos (email, teléfono, etc.)
\usepackage{hyperref}            % Links clickeables (email, LinkedIn)
\usepackage{enumitem}            % Control avanzado de listas

% --- Colores personalizados ---
\definecolor{primary}{RGB}{0, 90, 156}   % Azul UNLP para títulos
\definecolor{secondary}{RGB}{100, 100, 100} % Gris para texto secundario

% --- Estilo de secciones ---
\usepackage{titlesec}
\titleformat{\section}{\large\bfseries\color{primary}}{}{0em}{}[\titlerule]
\titlespacing*{\section}{0pt}{12pt}{6pt} % Espaciado: antes/después de sección

% --- Comandos personalizados ---
\newcommand{\cvitem}[2]{\textbf{#1} \hfill \color{secondary}#2} % Para fechas/lugares
\newcommand{\iconitem}[2]{\makebox[1.5em]{\fa#1} #2} % Ícono + texto (ej: \faEnvelope)

\begin{document}

% --- Encabezado ---
\begin{center}
    {\Huge \textbf{Lucentini Joaquín}} \\[15pt]
    
    \noindent
    \begin{tabular}{@{} l @{}}
    \faEnvelope{} joacolucen96@gmail.com \\
    \faMapMarker{} La Plata, Buenos Aires \\
    \faLinkedin{} \href{https://www.linkedin.com/in/joaquin-lucentini-a48066277/}{linkedin.com/Joaquin-Lucentini} \\
    \faGithub{} \href{https://github.com/JoacoLucen}{github.com/Joaquin-Lucentini}
    \end{tabular}
\end{center}

% --- Sección: Perfil ---
\section*{Perfil}
Estudiante de \textbf{Ciencia de Datos en Organizaciones} en la \textbf{Universidad Nacional de La Plata} (UNLP), actualmente me encuentro cursando el \textbf{2do año} de la carrera (ingreso en 2024). Interesado en machine learning, visualización, análisis y manejo de grandes volumenes de datos, trabajo en equipo, superación personal. Busco oportunidades para aplicar conocimientos académicos en proyectos reales.

% --- Sección: Formación Académica ---
\section*{Formación Académica}
\begin{itemize}[leftmargin=*]
    \item \cvitem{Ciencia de Datos en Organizaciones}{UNLP | 2024--Presente}
    \begin{itemize}
        \item Promedio: 8.6/10.00
    \end{itemize}
    \item \cvitem{Bachiller en Ciencias Naturales}{Instituto Niño Jesus | 2015--2020}
\end{itemize}

% --- Sección: Experiencia Laboral ---
%\section*{Experiencia Laboral}
%\begin{itemize}[leftmargin=*]
    %\item \cvitem{Auxiliar administrativo}{Centro Universitario Regional Saladillo | Feb 2023 -- Dic 2024}
    %\item \cvitem{Auxiliar administrativo}{Pro-Feed Saladillo | Trabajo de verano}
    % Añade más items según necesites
%\end{itemize}

% --- Sección: Habilidades Técnicas ---
\section*{Skills}
\begin{itemize}[leftmargin=*]
    \item \textbf{Lenguajes}: Python (intermedio)
    \item \textbf{Herramientas}: Jupyter, Git, Excel, Pandas, Matplotlib, Streamlit, Google Sheets, Google Finance
    \item \textbf{Idiomas}: Inglés (basico)
\end{itemize}

% --- Sección: Proyectos ---
\section*{Proyectos}
\begin{itemize}[leftmargin=*]
    \item \cvitem{Limpieza y visualización de datos del INDEC}{Python | 2025}
    \begin{itemize}
        \item Utilizacion de la EPH para Limpieza y Visualizacion de Datos: Utilice Python para la limpieza y visualizacion de la EPH (Encuesta Permanente de Hogares) y mostre los resultados, de una forma interactiva, usando Streamlit.
        \item Objetivo: Aplicar mis conocimientos y estudios para crear de forma practica y simple indicadores demograficos, mediante objetos graficos.
        \item \href{https://github.com/JoacoLucen/EPH-Insight-App}{Proyecto en GitHub}
        \item \href{https://eph-insight-app-joacolucentini.streamlit.app/}{Proyecto en la web}
    \end{itemize}
    \item \cvitem{Planilla de Inversiones Personales}{Google Sheets | 2025}
    \begin{itemize}
        \item Utilizacion de Google Sheets y Google Finance para crear un portfolio: Use herramientas como Google Sheets y Google Finance para crear una plantilla que permitiera tener un control de un portfolio de inversiones.
        \item Objetivo: Generar una plantilla que ayude a las personas a controlar sus activos financieros. Obteniendo datos pasados y presentes, gastos, ingresos, promedios, y rentabilidad, bajo un modelo FIFO (primero entrado - primero salido) que ayude a controlar de mejor forma el portfolio.
        \item \href{https://docs.google.com/spreadsheets/d/1MOAbafLv-NISA2nb2gy0AmwRGHegZyXV7TNal1StVgE/edit?usp=sharing}{Link a la Planilla}
    \end{itemize}
    \cvitem { }
    \item \cvitem{Escreapeo, Limpieza, Visualizacion y Analisis de Datos}{Python | 2025}
    \begin{itemize}
        \item Utilice Python y librerias como Requests, Beautifulsoup4, Streamlit, Plotly, Reportlab, entre otras para obtener datos de diferentes portales de noticias como TN, C5N, La Nacion y Clarin. Luego realice una limpieza de datos y su respectiva visualizacion en Streamlit y en un archivo PDF.
        \item Objetivo: 
        - Generar un programa que pueda obtener titulos de Internet para realizar un analisis politico, economico y social sobre cuanto habla cada medio de cada tema.
        - Realizar una limpieza de datos provenientes de internet, visualizarlos en Streamlit y generar un archivo PDF con un texto cambiante segun los datos y con imagenes correspondientes a los graficos de Streamlit.
        \item \href{https://github.com/JoacoLucen/scraping_web}{Proyecto en GitHub}
    \end{itemize}
\end{itemize}

% --- Sección: Cursos ---
%\section*{Cursos Certificados}
%\begin{itemize}[leftmargin=*]
    %\item \cvitem{ejemplo1}{Coursera | 2023}
    %\item \cvitem{ejemplo2}{Udemy | 2024}
%\end{itemize}

\end{document}